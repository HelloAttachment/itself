\chapter{The Geometry of Conceptualisation: Analogies} \label{chap:analogy}
\begin{table}
\begin{tabular}{lrrrrrr}
\hline
\emph{dimensions} & 5 & 10 & 20 & 50 & 100 & 200 \\
\hline
\parbox[t]{2mm}{\multirow{4}{*}{\rotatebox[origin=c]{90}{2x2}}} & \textsc{joint} & 0.911 & 0.972 & 0.989 & 0.986 & 0.970 & 0.916 \\
& \textsc{indy} & 0.000 & 0.000 & 0.001 & 0.090 & 0.356 & 0.677 \\
& \textsc{zipped} & 0.921 & 0.975 & 0.991 & 0.987 & 0.970 & 0.919 \\
\hline
\parbox[t]{2mm}{\multirow{4}{*}{\rotatebox[origin=c]{90}{\textsc{5x5}}}} & 0.941 & 0.987 & 0.996 & 0.997 & 0.957 \\
& \textsc{indy} & 0.000 & 0.000 & 0.012 & 0.098 & 0.202 & 0.610 \\
& \textsc{zipped} & 0.934 & 0.987 & 0.999 & 0.998 & 0.997 & 0.968 \\
\hline
\end{tabular}
\caption[Finding Spaces for Known Analogies]{Accuracy rates for solving analogies when choosing subsets of optimal dimensions from 400 dimensional subspaces picked based taking the first three elements of each analogy as input.}
\end{table}

\chapter{The Geometry of Conceptualisation: Analogies} \label{chap:analogy}
\begin{table}
\begin{tabular}{lrrrrrr}
\hline
\emph{dimensions} & 5 & 10 & 20 & 50 & 100 & 200 \\
\hline
\parbox[t]{2mm}{\multirow{4}{*}{\rotatebox[origin=c]{90}{2x2}}} & \textsc{joint} & 0.654 & 0.814 & 0.896 & 0.930 & 0.881 & 0.466 \\
& \textsc{indy} 0.000 & 0.000 & 0.000 & 0.072 & 0.356 & 0.636 \\
& \textsc{zipped} & 0.616 & 0.806 & 0.892 & 0.929 & 0.887 & 0.489 \\
\hline
\parbox[t]{2mm}{\multirow{4}{*}{\rotatebox[origin=c]{90}{\textsc{5x5}}}} & 0.657 & 0.828 & 0.901 & 0.921 & 0.835 & 0.402 \\
& \textsc{indy} & 0.000 & 0.000 & 0.003 & 0.074 & 0.225 & 0.569 \\
& \textsc{zipped} & 0.589 & 0.790 & 0.888 & 0.915 & 0.876 & 0.418 \\
\hline
\end{tabular}
\caption[Finding Spaces for Fake Analogies]{Accuracy rates for solving randomly completed analogies when choosing subsets of optimal dimensions from 400 dimensional subspaces picked based taking the first three elements of each analogy as input.}
\end{table}

\section{A Note on the Data}
It must be mentioned that the data that has been analysed in this chapter is of a very specific character.  The analogies put together by the team at Google are populated by a high percentage of proper names, in particular place names and also currencies, demonyms, and the like.  This belies a particular view of language and indeed cognition which is at odds with the premise motivating the model described in this thesis, as outlined at the beginning of Chapter~\ref{chap:theory}.  Proper names are, as \cite{Russell} has pointed out, particular kinds of words with peculiar denotational properties in that they refer to specific and unique entities or correspondingly specific classes of entities.  This is not to say that they do not admit ambiguity -- \emph{Paris} is the name of, among other things, a classical character, and \emph{Berlin} the name of a 1980s new wave band -- but there tends to be a certain clarity of intent when these types of words are used.  These types of analogies are exemplary of cases where language coalesces into a relatively stable conceptual representation, and, notwithstanding cases of polysemy, it's arguably not particularly surprising that these relationships emerge as commensurable directions in a likewise stable representational space.

Furthermore, it is telling that the designers of the dataset have chosen to refer to the variety of analogy typified by \emph{slow:slowly::fast:quickly} as \emph{syntactic}.  With reference to \cite{Saussure} and more lately in the distributional semantic paradigm \cite{Sahlgren}, I would rather call this type of analogy \emph{syntagmatic}, in that 
