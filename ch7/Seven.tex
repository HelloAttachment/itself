\chapter{Conclusion}

If the phenomenal experience of reality -- the quality of the perception of a world of objects and processes -- is, as \cite{Clark} puts it, \emph{controlled hallucination}, the words too are a layer of the 

Word frequency is not a thing that is directly perceived: though I could make an informed guess, I do not actually have a sense of how often I've encountered the word \emph{the} recently, versus the word \emph{thundercloud}, versus the word \emph{fulgurate}.  Meaning, on the other hand, if we take XXX seriously, is directly perceived.

``the familiar physiognomy of a word, the feeling that it has taken up its meaning into itself,'' \citep[][p. 218]{Wittgenstein}

What I am ultimately seeking to set up here is the groundwork for the application of computational techniques to phenomenologically oriented cognitive models.  This is why I've attempted to make my methodology conversant with, for instance, \cite{Davidson1978} theory of metaphor, which is ultimately about the way that language gets outside of the portage of propositions about situations in the world and into the actual fabric of the experience of existing.
