\chapter{Author's publications}

% publications goes here

\subsection*{Journal papers}
\begin{enumerate}
  \item \textbf{Y. Gao}, X. Chen, Z. Ying and C. G. Parini, ``Design and Performance
Investigation of a Dual-element PIFA Array at 2.5 GHz for MIMO Terminals," \emph{IEEE Trans. on
Antennas and Propagation}. (Revising)

\end{enumerate}


\subsection*{Conference papers}


\begin{enumerate}
  \item \textbf{Y. Gao}, X. Chen, Z. Ying and C. G. Parini , ``Further Investigation of a
Dual-Element Diversity PIFA for MIMO Applications at 2.5 GHz Band," \emph{IEEE International
Symposium on Antennas and Propagation (AP-S)} , Honolulu, Hawaii, USA  June, 2007. (Accepted)
    
\end{enumerate}

\subsection*{Project report}
\begin{enumerate}
  \item X. Chen and \textbf{Y. Gao}, ``Final Report on Modelling of Difficult Environments,"
  Galileo Advance Concept project final report (GAC/EUT/DT/219), 9th June - 31 October 2007.
  
\end{enumerate}



\renewcommand{\thefigure}{B.\arabic{figure}}
\chapter{Solutions for the Examples in Chapter 2}
\label{examples_solutions}
\section{Example 1: Antenna Spacing Effect}

\textbf{Step 1:} Get the $H_{norm}H_{norm}^\dagger$, set $2{\pi}R/\lambda={\omega}_1$ and
$2{\pi}(\overline{R})/\lambda={\omega}_2$ ($\lambda$ is the wavelength), so
\begin{equation}
H_{norm} = \frac{1}{\sqrt 2 }\left( {{\begin{array}{*{20}c}
 {e^{ - j\omega _1 }} \hfill & {e^{ - j\omega _2 }} \hfill \\
 {e^{ - j\omega _2 }} \hfill & {e^{ - j\omega _1 }} \hfill \\
\end{array} }} \right)
\end{equation}
Note: $H_{norm}^\dagger=(\overline{H_{norm}})^T$ and $e^{j\theta}=cos\theta +jsin\theta$
\begin{equation}
\begin{array}{l}
 H_{norm}H_{norm}^\dagger = \frac{1}{2}\left( {{\begin{array}{*{20}c}
 {e^{ - j\omega _1 }} \hfill & {e^{ - j\omega _2 }} \hfill \\
 {e^{ - j\omega _2 }} \hfill & {e^{ - j\omega _1 }} \hfill \\
\end{array} }} \right)\left( {{\begin{array}{*{20}c}
 {e^{j\omega _1 }} \hfill & {e^{j\omega _2 }} \hfill \\
 {e^{j\omega _2 }} \hfill & {e^{j\omega _1 }} \hfill \\
\end{array} }} \right) \\
 = \frac{1}{2}\left( {{\begin{array}{*{20}c}
 {1 + 1} \hfill & {e^{ - j(\omega _1 - \omega _2 )} + e^{ - j(\omega _1 -
\omega _2 )}} \hfill \\
 {e^{ - j(\omega _1 - \omega _2 )} + e^{ - j(\omega _1 - \omega _2 )}}
\hfill & {1 + 1} \hfill \\
\end{array} }} \right) \\
 = \left( {{\begin{array}{*{20}c}
 1 \hfill & {\cos (\omega _1 - \omega _2 )} \hfill \\
 {\cos (\omega _1 - \omega _2 )} \hfill & 1 \hfill \\
\end{array} }} \right) \\
 \end{array}
\end{equation}
\textbf{Step 2:} Get the eigenvalue of $H_{norm}H_{norm}^\dagger$, here set $\lambda$ as
eigenvalue, then,
\begin{equation}
\left( {{\begin{array}{*{20}c}
 {1 - \lambda } \hfill & {\cos (\omega _1 - \omega _2 )} \hfill \\
 {\cos (\omega _1 - \omega _2 )} \hfill & {1 - \lambda } \hfill \\
\end{array} }} \right) = 0
\end{equation}
and $\omega_1-\omega_2=2{\pi}R/\lambda-2{\pi}(\overline{R})/\lambda=-54^o$, so
$\lambda=1{\pm}cos(\omega_1-\omega_2){\approx}1{\pm}0.59$. Hence, $\lambda_1=1.59$ and
$\lambda_2=0.41$.




