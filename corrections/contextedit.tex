\paragraph{A Note on \emph{Context}} \revAK{15}{The term \emph{context} has been used widely and varyingly by authors in both theoretical and computational linguistics, and with good reason, as various senses of the concept of context are clearly at play in any serious discussion of the interplay between language and cognition.  Statistically minded computational linguists in particular, of whom I would like to count myself as one, have often used \emph{context} to refer to the window of co-occurrence in which a word token is observed within a sample of text.  In his description of a co-occurrence statistic for measuring semantic similarity,} \cite{Schutze1992b} \revAK{15}{introduced the term \emph{context space} to refer to a space of co-occurrence dimensions, a terminology subsequently adopted by} \cite{BurgessEA1997} \revAK{15}{in relation to their HAL system.  This notion of proximity within a text as context has persevered in the natural language processing literature.}

\revAK{15}{Theoretical linguists and cognitive scientists, on the other hand, have tended to treat \emph{context} as a much more general condition wrapped up with the entire perceptual, phenomenological aspect of existing as a cognitive agent in a complex world.  So for instance} \citeauthor{Bateson1972} \revAK{15}{says that ``message material, or information, comes out of a context into a context,''} \cite[][p. 404]{Bateson1972}, \revAK{15}{meaning that there is an alignment between the inner context of an agent and the outer context of the world, while} \citepos{Grice1975} \revAK{15}{notion of \emph{implicature} holds that meaning is somehow always determined in a context, with the exact nature of context remaining somewhat open-ended, and this nomenclature has been carried on by subsequent researchers interested in the idea that cognition, conceptualisation, and, correspondingly, language are always in some way specified by a situation in the world.  The idea is that context is probably something that exists in large part outside of language, and almost certainly outside the informationally restrictive confines of word co-occurrences within a sentence.}

\cite{MillerEA1991} \revAK{15}{address this definitional vagary in the context of early work on distributional semantics, and specifically opt to use context to refer to the co-occurrences that occur on a purely lexical and sentential level.  In my thesis, which seeks to address both those components of language measurable by an information processing system and the more general question of meaning as an environmentally situated phenomenon, I will endeavour to use the term \emph{context} strictly in reference to the latter notion of the situation in which concepts and semantics emerge in tandem.  With regard to words observed in proximity to one another, on the other hand, I will refer to \emph{co-occurrence}, and so additionally to a \emph{co-occurrence window} within which such observations are made and correspondingly a \emph{co-occurrence statistic} as a measure of the relative frequency of such observations.  Hopefully this terminological committment will serve to avoid confusion.}
