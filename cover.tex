
\title{Stage Two Report:\\
A Geometric Method for Context Sensitive Distributional Semantics}

\author{Stephen McGregor}
\department{Department of Electronic Engineering} \college{Queen Mary, University of London}
\degree{Doctor of Philosophy} \degreemonth{July} \degreeyear{2017}

%% By default, the thesis will be copyrighted to MIT.  If you need to
%% copyright the thesis to yourself, just specify the `vi' documentstyle
%% option.  If for some reason you want to exactly specify the copyright
%% notice text, you can use the \copyrightnoticetext command.
%\copyrightnoticetext{\copyright ~University of London, 2006}

% The dedication info.
%\dedication{TO MY FAMILY}

% Make the titlepage based on the above information.  If you need something
% special and can't use the standard form, you can specify the exact text of
% the titlepage yourself.  Put it in a titlepage environment and leave blank
% lines where you want vertical space. The spaces will be adjusted to fill
% the entire page. The dotted lines for the signatures are made with the
% \signature command.

% Make the first title page
\maketitle

% make the dedication page
%\makededication

% Start to count page number from abstract page
\pagestyle{plain}%
\setcounter{page}{1}
\pagenumbering{roman} %

% The abstractpage environment sets up everything on the page except the
% text itself.
%
% You can either \input (*not* \include) your abstract file, or you can put
% the text of the abstract directly between the \begin{abstract} and
% \end{abstract} commands.
\begin{abstract}
%\input{abstract}
% abstract goes here
This thesis describes a novel methodology, grounded in the distributional semantic paradigm, for building context sensitive models of word meaning, affording an empirical exploration of the relationship between words and concepts. Anchored in theoretical linguistic insight regarding the contextually specified nature of lexical semantics, the work presented here explores a range of techniques for the selection of subspaces of word co- occurrence dimensions based on a statistical analysis of input terms as observed within large-scale textual corpora. The relationships between word-vectors that emerge in the projected subspaces can be analysed in terms of a mapping between their geometric features and their semantic properties. The power of this modelling technique is its ability to generate ad hoc semantic relationships in response to an extemporaneous linguistic or conceptual situation. 

The product of this approach is a generalisable computational linguistic methodology, capable of taking input in various forms, including word groupings and sentential context, and dynamically generating output from a broad base model of word co-occurrence data.  To demonstrate the versatility of the method, this thesis will present competitive empirical results on a range of established natural language tasks including word similarity and relatedness, metaphor and metonymy detection, and analogy completion. A range of techniques will be applied in order to explore the ways in which different aspects of projected geometries can be mapped to different semantic relationships, allowing for the discovery of a range of lexical and conceptual properties for any given input and providing a basis for an empirical exploration of distinctions between the semantic phenomena under analysis. The case made here is that the flexibility of these models and their ability to extend output to evaluations of unattested linguistic relationships constitutes the groundwork for a method for the extrapolation of dynamic conceptual relationships from large-scale textual corpora. 

This method is presented as a complement and a counterpoint to established distributional methods for generating lexically productive word-vectors. Where contemporary vector space models of distributional semantics have almost universally involved either the factorisation of co-occurrence matrices or the incremental learning of abstract representations using neural networks, the approach described in this thesis preserves the connection between the individual dimensions of word-vectors and statistics pertaining to observations in a textual corpus. The hypothesis tested here is that the maintenance of actual, interpretable information about underlying linguistic data allows for the contextual selection of non-normalised subspaces with more nuanced geometric features. In addition to presenting competitive results for various computational linguistic targets, the thesis will suggest that the transparency of its representations indicates scope for the application of this model to various real-world problems where an interpretable relationship betweendata and output is highly desirable. This, finally, demonstrates a way towards the productive application of the theory and philosophy of language to computational linguistic practice.

\end{abstract}

% Acknowledgments
%
% You can either \input (*not* \include) your acknowledgments file, or you can put
% the text of the acknowledgments directly between the \begin{acknowledgments} and
% \end{acknowledgments} commands.
%\begin{acknowledgments}
%\input{acknowledgments}
%Acknowledgment goes here...

%\end{acknowledgments}

%\begin{context}
%The term \emph{context} has been used widely and variously by authors in both theoretical and computational linguistics, and with good reason, as various sense of the concept of context are clearly at play in any serious discussion of the interplay between language and cognition.  Statistically minded computational linguists in particular, of whom I would like to count myself as one, have often used \emph{context} to refer to the window of co-occurrence in which a word token is observed within a sample of text.  In his description of a co-occurrence statistic for measuring semantic similarity, \cite{Salton1992b} introduced the term \emph{context space} to refer to a space of co-occurrence dimensions, a terminology subsequently adopted by \cite{BurgessEA1997} in relation to their HAL system.  This notion of proximity within a text as context has persevered in the natural language processing literature.

%Theoretical linguists and cognitive scientists, on the other hand, have tended to treat \emph{context} as a thing 

%So 

%and this nomenclature has been carried on by subsequent researchers interested in the idea that cognition, conceptualisation, and, correspondingly, language are always in some way specified by a situation in the world.

%In this thesis, I will endeavour to use the term \emph{context} strictly in reference to the latter notion of 

%\end{context}

\begin{glossary}
\begin{description}
\item[base space] A high dimensional, sparse vector space of word-vectors, delineated in terms of dimensions of co-occurrence statistics.
\item[context] The situation -- environmental, cognitive, perceptual, linguistic, and otherwise -- in which an agent finds itself and applies language to meaning.
\item[contextual input] A set of words characteristic of a category or semantic relationship used to generate a subspace for the modelling of semantic phenomena.
\item[dimension selection] The process of contextually choosing a subset of dimensions in order to project a subspace from a base space.
\item[co-occurrence] The observation of one word in proximity to another in a corpus.
\item[co-occurrence statistic] A measure of the tendency for one word to be observed in proximity to another across a corpus.
\item[co-occurrence window] The boundary defining the proximity within which two words are considered to be co-occurring, typically a distance in terms of words within a sentence.
\item[methodology] The process of building base spaces from observations of co-occurrences within a corpus and contextually projecting subspaces through dimension selection.
\item[model] An application of methodology to a particular linguistic task or experiment, sometimes including task specific statistical analysis techniques.
\item [subspace] A context specific lower-dimensional projection from a base space, effectively mapping semantic relationships to a context by way of the geometric relationships between word-vectors.
\item[word-vector] A high-dimensional geometrically situated semantic representation of a word, constructed as an array of co-occurrence statistics.
\end{description}
\end{glossary}
