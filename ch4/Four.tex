\chapter{Conceptual Clusterings, Similarity, and Relatedness}
In Chapter~\ref{chap:theory}, I laid out the theoretical groundwork for statistical context sensitive models of lexical semantics, and in Chapter~\ref{chap:method} I described the actual methodology for building such much.  In this chapter, I will now present the first set of experiments designed to evaluate the utility of this methodology.  These experiments are intended to probe the productivity of a context sensitive, geometric approach to building a computational model of semantics based on statistics about word co-occurrences.  They encompass two different experimental set-ups and corresponding varieties of data, one of which has been designed specifically for the purpose of testing my ideas and one of which involves an assortment of data used pervasively by computational linguistics interested in semantic models.

The first experiment, presented as a proof of concept, involves using multi-word phrases as input and evaluating the methodology's capacity for building subspaces where words associated with the conceptual category denoted by the input term can be reliably discovered.  This experiment expands upon the notion of proto-conceptual spaces outlined in the previous chapter, considering whether the word vectors that populate regions of subspaces are characterised by a certain categorical coherence.  In the case of the data explored here, the experiment is specifically set up to feel out the contextual capacity of my methodology and compare it to a standard generic semantic space.  The question asked is whether the shifts from subspace to subspace based on particular input yield productive alterations in the way that words both cluster and emerge from the melange of word-vectors that circulate around my base model.

The second experiment moves into more familiar computational linguistic territory, using some well-travelled datasets to examine the methodology's capacity for identifying two related but distinct semantic phenomena: relatedness and similarity.  Each of these objectives have provided reliable but distinct evaluative criteria for computational models of lexical semantics.  One of the hypotheses I will put forward regarding my methodology is that the geometrically replete subspaces generated by my contextualisation techniques should provide features for the simultaneous representation of related, diverse, and sometimes antagonistic aspects of language.  Experimenting with these established datasets will provide a platform for exploring the ways in which different features of a semantic structure projected into one of my contextualised subspaces shift as the relationships inherent in the generation of the subspace likewise change, and this will in turn lead to some searching questions about the importance of context in the computational modelling of these particular semantic phenomena in the first place.

\section{A Proof of Concept}
In this section, I present the first experiment performed using my contextually dynamic distributional semantic model.  The gist of this experiment is to take a word pair representing a compound noun -- for instance, \emph{body part} -- and see if my methodology can use the word pair to contextually generate a space where other words conceptually related to that compound noun can be found in a systematic way.  This is conceived of as an entailment task, in that I will attempt to find phrases considered to be categorical constituents of the concept represented by the word pair, taking the WordNet lexical taxonomy as a ground truth.  There is a scholastic back story here.

An early version of this experiment was reported in \cite{AgresEA2015}.  That first effort arose out of a question posed by a colleague regarding the feasibility of using a statical NLP technique for generating categorical labels that could be used to evaluate computational creativity in a domain specific way \citep[for a psychological perspective on the difficulty of generating such terms in an objective way using human subjects, see][]{VanDerVeldeEA2015}.  So, for instance, given a creative domain such as \textsc{musical creativity}, could a distributional semantic model generate terms that are reliably relevant to the concept denoted by that phrase, rather than the potentially disparate properties independently associated with \textsc{music} and \textsc{creativity}?  Intuitively there seems to be little reason to hope that the space halfway between these points in a general semantic space would somehow adequately represent the properties of the overall concept.  The early work explored the dimensions contextually selected by analysing the co-occurrence features of word-vectors corresponding to inputs along the lines of the expository results presented anecdotally in Chapter~\ref{chap:method}, but without any rigorous evaluation.

Reviewer responses to a subsequent journal article \citep{McGregorEA2015c}, designed as a more thorough introduction of the methodology, inspired a computationally oriented mode of evaluation.  The experiment that has emerged involves attempting to recapitulate taxonomical conceptual relationships from the WordNet database \citep{Fellbaum1998}.  Wordnet is a lexical taxonomy of \emph{synsets}, basically semantic word senses, arranged into a hierarchy of entailment relationships, with each synset associate with a number of \emph{lemmas}, word types indexed by that synset according to human annotators.  This experiment takes as input instances of synsets labelled by compound noun phrases and seeks to output as many of the lemmas listed associated with synsets that are hyponyms of the input synset.  So, for instance, if there were a synset labelled \textsc{wild animals}, words such as \emph{lion}, \emph{tiger}, and \emph{bear} would presumably be positive outputs based on that input.\footnote{In keeping with the convention used elsewhere in this thesis, synset labels will be presented in small caps and lemmas will be presented in italics.}

To this end, 12 of the top synset labels consisting of compound noun phrases are extracted from WordNet.  These labels are chosen such that the 12 most populous, in terms of unique lemmas associated with hyponym synsets, are included, with the constraint that no synset can be a hypernym of another one of the target synsets.  All lemmas associated with all hyponyms of each synset are extracted and grouped.  The terms labelling a given synset are then passed to my model, with the corresponding word-vectors serving as the basis for dimensional selection using the \textsc{joint}, \textsc{indy}, and \textsc{zipped} techniques as outlined in Chapter~\ref{chap:method}.  The subspaces returned by each of these techniques are explored to return the top terms using both of the procedures outlined in Chapter~\ref{sec:twomeasures}: the terms closes to the mean point between the input word-vectors in a subspace are returned, and the terms furthest from the origin -- the terms with the largest norm -- in a given subspace are returned.  The experimental vocabulary is considered to be the intersection of the list of all WordNet lemmas with the vocabulary of my model (the 200,000 most frequent word types in Wikipedia).

