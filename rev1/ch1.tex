\chapter{Introduction}
In Italo \citepos{Calvino1983} whimsical novel of ideas \emph{Mr. Palomar}, the titular character reflects upon the mating rituals of turtles.  Encased as they are in an unfeeling shell, with only a filament of sensing flesh with which to prod the world, Palomar concludes that for turtles sensuality must take on an almost entirely cerebral aspect: ``The poverty of their sensorial stimuli perhaps drives them to a concentrated, intense mental life, leads them to a crystalline inner awareness,'' (p. 18).

The chelonian condition imagined by Palomar is, at least in a certain metaphoric respect, the endemic situation of a statistical language model.  The environment of such a model is at best a sparse simulacrum of existence, a trickle of information that is only about anything in a sense completely external to the model itself.  If language is characterised by an aboutness that anchors itself in acts, intentions, and experiences in the world, then the blind traversal of corpora severs the semantic anchor lines and allows a model to curl into itself.  The residue of this process, an accumulation of symbols trailed by kite tails of numbers, can really only be directly interpreted as an echo of language, revealing tendencies but demanding clever interpretive reconstruction in order to be of any practical use.

This thesis sets out to explore the question of if and how a statistical language model can be involved in the generation of new meaning.  That such models can at least be interpreted in a way that compellingly exposes semantic and syntactic relationships has been demonstrated, and the trend in ongoing work in this area is to continue to develop mechanisms for recapitulating datasets while allowing the features of the models themselves to drift further away from semantic transparency.  The positive outcome is a battery of computationally tractable and sometimes mathematically elaborate mechanisms for processing natural language, some of which appear to have real practical applications in tasks such as information retrieval and machine translation.  There is a risk, however, of a proliferation of models that are very good at solving bespoke micro-problems but reveal very little about the nature of language.

One of the objectives of this work, then, is to explore the relationship between the abstract geometry of relatively high-dimensional language models and the concrete geometry of the world that language must be about.  \cite{Kant} places the geometric representation of the world at the foundation of mental existence, classing it in terms of fundamental \emph{intuitions} rather than supervenient \emph{concepts}.  Already a parallel between language and conceptualisation is emerging here: concepts are about space, and also somehow composed of space, in the way that words are the world and also in the world.

When words become points, the language itself, consisting of those words and the relationships between them, becomes a spatial entity, but in an abstract space where there is not even a nominal distinction between space and information.  It is no longer a space in the essential sense of Kant, but rather a generative space, a space that is somehow an index to that other space which is co-extensive with reality.  In this kind of space, it is the space itself that does the work: it is the space that comes to characterise the features of each point, and, when points interact, they interact by virtue of nothing other than their positions.  When a statistical language model is interpreted, it is therefore the internal space of the web of relationships working as a dynamic system unto itself that are being explored.

This thesis seeks to ground these philosophical arguments in computational experimentation.  The methodology described and pursued in the following pages finds its motivation in theoretical considerations of language and cognition, but its techniques are rooted in the contemporary approach to statistical language modelling.  The unique contribution of the model that will be described here, though, is to offer a robust mechanism for the prolific contextualisation of the lexical information inherent in the network of statistics that delineate its spatial aspect.  In this regard, what is being modelled is the relationship between language and concepts as they come about in the world, in the unfolding, ready-to-hand, non-sentential character of cognition.

With that said, no strong claim will be made here about having resolved the hard questions surrounding the mind.  There is no suggestion that the model described in this thesis is somehow simulating cognition; indeed, the mechanism of a computational model such as this, substantiated by nothing more than the drizzle of data that penetrates its shell, is specifically different than the way a cognitive agent evolves into a situation of deep, tight, multifarious entanglement with the world.  Furthermore, it is essential not simply to dismiss the challenges that face any attempt at a philosophically robust description of cognition.  Indeed, these philosophical problems will become a guiding light for this work, with the hope that a good model that is honest about the level of abstraction at which it is applicable may reveal some interesting aspect of the elaborate whole.

The strong claim which will be made is that the model described here can participate autonomously in the discovery of new and useful meaning, and it is in this regard that this work finds its roots in the multifaceted field of computational creativity.  When \cite{Wittgenstein} states that ``only the act of meaning can anticipate reality'' (p. 76), he suggests that beyond a mere syntactic encoding of events in the world, language is tied up in a process of connecting mental existence to being in the world.  Creativity, construed as meaning making, is taken as a broad target for a project that attempts to model the semantics inherent in the relationship between words and concepts.  The work presented here does not aspire to the kind of phenomenological richness that is at the heart of Wittgenstein's later philosophy, but it has been designed to generate informational structures that might be interpreted as models of internal representations with the interactional dynamics necessary for generating significant new meaning.

So, while there may be a popular perception that figurative language is in some regard more creative than plainly propositional language, the choice of metaphor as a target is actually motivated by the way in which figurative language exposes the functionality of words.  In the course of constructing a metaphor, a linguistic agent is seeing meaning as an affordance for some communicative action: just as a shoe might become a hammer, or a chair a weapon, the concept of a butcher comes to stand in for a certain type of surgeon in a certain context.  It is this opportunism inherent in language which the project presented here will attempt to model.

\section{A Hypothesis}
The premise of this this thesis is that conceptualisation can be modelled in geometric terms, not least because concepts are about a reality that is essentially geometric.  Furthermore, the relationship between concepts and language is to be discovered in the way that language is in the world: language is about the world, but it is also made of the world, and the relationship between the two is as determinate as it is irretrievable.  When language comes about as an affordance of communication, it is a physical opportunity for expressive action that is being directly perceived as an aspect of an environment, as a way to use and change that environment.  The situation of language is inextricably spatial and functional, and so the language itself can only really properly be modelled as something that is dynamic and geometric.

Following on this, the hypothesis offered here is that figurative language can be modelled in terms of the geometrical properties of a statistical language model.  In particular, this paper predicts that a dynamically generative distributional semantic model, equipped with a facility for projecting contextually informed subspaces from a base lexical space, will remit clusterings of word-vectors that can be interpreted as geometric representations of conceptualisations, and the work to be described in the following chapters is predicated upon this prediction.  A significant consequence of the geometric modelling of concepts is that the geometric properties of these conceptually charged word spaces will allow for the isomorphic mapping between conceptual representations as a methodology for modelling the generation of figurative language based on contextually specific information.

By way of testing this hypothesis, a statistical language model will be developed and subjected to a set of experiments.  In particular, the model will be tested in terms of established approaches to the recapitulation and extension of lexical ontologies and to the completion of analogies.  Furthermore, in an effort to test the pragmatic efficacy of the model, metaphoric output will be measured in a study involving human subjects and also, through a social media application of a metaphor generating bot, submitted to the tumult of public discourse.  The purpose of this final measure is to test the pragmatic applicability of the model's output in a real communicative situation.  It is by virtue of this last point in particular that this remains fundamentally a thesis about computational creativity, in that the trafficking of statements produced by an on-line application will be taken as affirmation that the behaviour of a system associated with the model has been interpreted as essentially creative in its generation of new, useful language.

\section{Contributions to the Field}
The concrete manifestation of the work described here is a system for generating conceptual metaphors: given linguistic input describing a basic situation in the world, for instance, ``that surgeon is sloppy'', the system will return a figuratively loaded description such as ``that surgeon is a butcher''.  The mechanism involved in discovering such new relationships between conceptual entities employs dynamically interactive representations, in that the mathematical, geometric nature of the model's word-labelled representations provides a platform for intermeshings which might lead to the emergence of representations that are dynamic on a conceptual level.  As such, from the perspective of computational creativity, there is a case to be made that something like internal representations of conceptual schemes are being modelled, and it will be argued here that this model can be considered in terms of the autonomous generation of novel, compelling linguistic artefacts through a process that can be deemed creative.

A similarly concrete but broader contribution of this work is the presentation of a new language model, specifically engineered to handle the nuance of conceptualisation that the pragmatics of natural language capture so well.  More than just another new model for performing a range of NLP type tasks, the goal here is to provide an endorsement for a particular type of model, namely, one that maintains features that are, in a statistical sense, interpretable as actual informational observations.  As \cite{White2009} has put it, ``we need some sort of cognitive grasp'' (p. 99) on the way in which statistical techniques tease a semantic model out of large scale corpora.  The work presented here is designed as a transparent implementation of a theory about the contextualisation involved in mapping from words to concepts, and so the expectation is that this might at least indicate a methodological stance for achieving this cognitive desideratum.

Finally, on a more abstract level, this work is offered as an indication of one way in which computers might be used to empirically demonstrate the efficacy of philosophical arguments.  In particular, the model presented here takes a theory about the nature of language and the contextual way in which language becomes deeply entangled with conceptualisation, and it applies this theory to an exercise in the design and testing of a model.  One of the broad aspirations of this dissertation is to hint at a way in which future work in both computer science and philosophy can seek a harmonious balance between theory and practice.  This is offered as something of an antidote to the fraught relationship between the philosophy of mind and the application of computer modelling, which has been persistently hampered by very reasonable objections to the erroneous claim that minds and computers are somehow just like each other.  The counterargument offered here is that minds and computers are fundamentally different, but there is still much to be gained in terms of both insight and application from using computers to model some of the things -- language, conceptualisation, creativity -- that are clearly native to the mind.  The key here is to keep in mind certain truths about the physical and, on the other hand, observer dependent nature of computing, and not to lose sight of the level of abstraction at which computational models can be understood.

\section{Structure of The Dissertation}
In addition to this introductory chapter, this thesis contains six more chapters.  The second chapter will present a review of relevant literature, both in terms of historical theoretical material and contemporary empirical work in the several fields related to this project.  This literature review will be followed by an exposition of the theoretical foundation of the project itself, describing the conceptual basis for the experimental work that will subsequently be described.

The fourth chapter will lay out the methodological approach to the construction of a generative language model, as well as the equations describing the mechanism for using this base model to project contextually informed subspaces.  It will also describe techniques for taxonomy recapitulation and the discovery of conceptual metaphors by way of mapping between word clusters discovered in the base model's projections.  The fifth chapter will describe the implementation of the language model, including its application to tasks involving completion of established test sets and recapitulation of widely used lexical ontologies.  Results of these experiments will also be presented in this chapter.

The sixth chapter will discuss the experimental results presented in the previous chapter from an analytical perspective.  Additionally, this chapter will present two further mechanisms for evaluating the model's metaphor generating capabilities in particular, in terms of a study involving human subjects and a pragmatically geared application of social media.  The seventh and final chapter will provide a holistic diagnosis of the project, returning in particular to the philosophical ambitions of this work and also considering potential future applications of the model.
